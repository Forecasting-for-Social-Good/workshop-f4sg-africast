% Options for packages loaded elsewhere
\PassOptionsToPackage{unicode}{hyperref}
\PassOptionsToPackage{hyphens}{url}
%
\documentclass[
  14pt,
  ignorenonframetext,
  aspectratio=169,
]{beamer}
\usepackage{pgfpages}
\setbeamertemplate{caption}[numbered]
\setbeamertemplate{caption label separator}{: }
\setbeamercolor{caption name}{fg=normal text.fg}
\beamertemplatenavigationsymbolsempty
% Prevent slide breaks in the middle of a paragraph
\widowpenalties 1 10000
\raggedbottom

\usepackage{amsmath,amssymb}
\usepackage{iftex}
\ifPDFTeX
  \usepackage[T1]{fontenc}
  \usepackage[utf8]{inputenc}
  \usepackage{textcomp} % provide euro and other symbols
\else % if luatex or xetex
  \usepackage{unicode-math}
  \defaultfontfeatures{Scale=MatchLowercase}
  \defaultfontfeatures[\rmfamily]{Ligatures=TeX,Scale=1}
\fi
\usepackage{lmodern}
\usetheme[]{monash}
\ifPDFTeX\else  
    % xetex/luatex font selection
\fi
% Use upquote if available, for straight quotes in verbatim environments
\IfFileExists{upquote.sty}{\usepackage{upquote}}{}
\IfFileExists{microtype.sty}{% use microtype if available
  \usepackage[]{microtype}
  \UseMicrotypeSet[protrusion]{basicmath} % disable protrusion for tt fonts
}{}
\makeatletter
\@ifundefined{KOMAClassName}{% if non-KOMA class
  \IfFileExists{parskip.sty}{%
    \usepackage{parskip}
  }{% else
    \setlength{\parindent}{0pt}
    \setlength{\parskip}{6pt plus 2pt minus 1pt}}
}{% if KOMA class
  \KOMAoptions{parskip=half}}
\makeatother
\usepackage{xcolor}
\newif\ifbibliography
\setlength{\emergencystretch}{3em} % prevent overfull lines
\setcounter{secnumdepth}{-\maxdimen} % remove section numbering

\usepackage{color}
\usepackage{fancyvrb}
\newcommand{\VerbBar}{|}
\newcommand{\VERB}{\Verb[commandchars=\\\{\}]}
\DefineVerbatimEnvironment{Highlighting}{Verbatim}{commandchars=\\\{\}}
% Add ',fontsize=\small' for more characters per line
\usepackage{framed}
\definecolor{shadecolor}{RGB}{248,248,248}
\newenvironment{Shaded}{\begin{snugshade}}{\end{snugshade}}
\newcommand{\AlertTok}[1]{\textcolor[rgb]{0.94,0.16,0.16}{#1}}
\newcommand{\AnnotationTok}[1]{\textcolor[rgb]{0.56,0.35,0.01}{\textbf{\textit{#1}}}}
\newcommand{\AttributeTok}[1]{\textcolor[rgb]{0.77,0.63,0.00}{#1}}
\newcommand{\BaseNTok}[1]{\textcolor[rgb]{0.00,0.00,0.81}{#1}}
\newcommand{\BuiltInTok}[1]{\textcolor[rgb]{0.00,0.00,0.00}{#1}}
\newcommand{\CharTok}[1]{\textcolor[rgb]{0.31,0.60,0.02}{#1}}
\newcommand{\CommentTok}[1]{\textcolor[rgb]{0.56,0.35,0.01}{\textit{#1}}}
\newcommand{\CommentVarTok}[1]{\textcolor[rgb]{0.56,0.35,0.01}{\textbf{\textit{#1}}}}
\newcommand{\ConstantTok}[1]{\textcolor[rgb]{0.00,0.00,0.00}{#1}}
\newcommand{\ControlFlowTok}[1]{\textcolor[rgb]{0.13,0.29,0.53}{\textbf{#1}}}
\newcommand{\DataTypeTok}[1]{\textcolor[rgb]{0.13,0.29,0.53}{#1}}
\newcommand{\DecValTok}[1]{\textcolor[rgb]{0.00,0.00,0.81}{#1}}
\newcommand{\DocumentationTok}[1]{\textcolor[rgb]{0.56,0.35,0.01}{\textbf{\textit{#1}}}}
\newcommand{\ErrorTok}[1]{\textcolor[rgb]{0.64,0.00,0.00}{\textbf{#1}}}
\newcommand{\ExtensionTok}[1]{\textcolor[rgb]{0.00,0.00,0.00}{#1}}
\newcommand{\FloatTok}[1]{\textcolor[rgb]{0.00,0.00,0.81}{#1}}
\newcommand{\FunctionTok}[1]{\textcolor[rgb]{0.00,0.00,0.00}{#1}}
\newcommand{\ImportTok}[1]{\textcolor[rgb]{0.00,0.00,0.00}{#1}}
\newcommand{\InformationTok}[1]{\textcolor[rgb]{0.56,0.35,0.01}{\textbf{\textit{#1}}}}
\newcommand{\KeywordTok}[1]{\textcolor[rgb]{0.13,0.29,0.53}{\textbf{#1}}}
\newcommand{\NormalTok}[1]{\textcolor[rgb]{0.00,0.00,0.00}{#1}}
\newcommand{\OperatorTok}[1]{\textcolor[rgb]{0.81,0.36,0.00}{\textbf{#1}}}
\newcommand{\OtherTok}[1]{\textcolor[rgb]{0.56,0.35,0.01}{#1}}
\newcommand{\PreprocessorTok}[1]{\textcolor[rgb]{0.56,0.35,0.01}{\textit{#1}}}
\newcommand{\RegionMarkerTok}[1]{\textcolor[rgb]{0.00,0.00,0.00}{#1}}
\newcommand{\SpecialCharTok}[1]{\textcolor[rgb]{0.00,0.00,0.00}{#1}}
\newcommand{\SpecialStringTok}[1]{\textcolor[rgb]{0.31,0.60,0.02}{#1}}
\newcommand{\StringTok}[1]{\textcolor[rgb]{0.31,0.60,0.02}{#1}}
\newcommand{\VariableTok}[1]{\textcolor[rgb]{0.00,0.00,0.00}{#1}}
\newcommand{\VerbatimStringTok}[1]{\textcolor[rgb]{0.31,0.60,0.02}{#1}}
\newcommand{\WarningTok}[1]{\textcolor[rgb]{0.56,0.35,0.01}{\textbf{\textit{#1}}}}

\providecommand{\tightlist}{%
  \setlength{\itemsep}{0pt}\setlength{\parskip}{0pt}}\usepackage{longtable,booktabs,array}
\usepackage{calc} % for calculating minipage widths
\usepackage{caption}
% Make caption package work with longtable
\makeatletter
\def\fnum@table{\tablename~\thetable}
\makeatother
\usepackage{graphicx}
\makeatletter
\def\maxwidth{\ifdim\Gin@nat@width>\linewidth\linewidth\else\Gin@nat@width\fi}
\def\maxheight{\ifdim\Gin@nat@height>\textheight\textheight\else\Gin@nat@height\fi}
\makeatother
% Scale images if necessary, so that they will not overflow the page
% margins by default, and it is still possible to overwrite the defaults
% using explicit options in \includegraphics[width, height, ...]{}
\setkeys{Gin}{width=\maxwidth,height=\maxheight,keepaspectratio}
% Set default figure placement to htbp
\makeatletter
\def\fps@figure{htbp}
\makeatother

% Colors
\definecolor{shadecolor}{RGB}{225,225,225}
\setbeamercolor{description item}{fg=Orange}
\definecolor{MonashBlue}{RGB}{66,109,152}
\definecolor{burntorange}{rgb}{0.8, 0.33, 0.0}

% Packages
\usepackage{amsmath, bm, amssymb, amsthm, mathrsfs,pifont}
\usepackage{url}
\usepackage{multirow, booktabs, float, textcmds, siunitx}
\usepackage{bm,booktabs,animate,ragged2e,multicol,microtype,hyperref}
\usepackage{fontawesome5}

% Figures
\graphicspath{{figs/}}

% Fonts
\fontsize{13}{15}\sf
\setbeamerfont{title}{series=\bfseries,parent=structure,size=\fontsize{24}{24}}
\ifcsname Shaded\endcsname
  \definecolor{shadecolor}{RGB}{225,225,225}
  \renewenvironment{Shaded}{\vspace*{0.15cm}\color{black}\fontsize{10}{10}\sf\begin{snugshade}\color{black}}{\end{snugshade}}
\fi

% gt dependencies
\usepackage{longtable,caption,setspace}
\captionsetup{font={small,stretch=0.80}}

% My definitions

\def\E{\text{E}}
\def\V{\text{Var}}
\def\up#1{\raisebox{-0.3cm}{#1}}
\def\pred#1#2#3{\hat{#1}_{#2|#3}}
\def\damped{$_\text{d}$}
\def\h+{h_{m}^{+}}
\def\str#1{\rlap{#1}\textcolor{red}{\rule{1cm}{0.1cm}}}

\def\st#1{\rlap{#1}\textcolor{red}{\rule{1cm}{0.1cm}}}
\def\bY{\bm{y}}
\def\by{\bm{y}}
\def\bS{\bm{S}}
\def\bI{\text{\rm\textbf{I}}}
\def\bbeta{\bm{\beta}}
\def\bSigma{\bm{\Sigma}}
\def\bW{\bm{\Sigma}}
\def\Var{\text{Var}}
\def\var{\text{Var}}
\def\bOmega{\bm{\Omega}}
\def\bLambda{\bm{\Lambda}}
\let\mc\multicolumn
\def\hl{\color[RGB]{230, 172, 0}}

\def\forecast{\begin{alertblock}{}\fontsize{10}{11}\sf
 A forecast is an estimate of the probabilities of possible futures.
\end{alertblock}}

\def\simfutures{\begin{textblock}{2.9}(12.8,7.7)
\begin{block}{}\fontsize{10}{11}\sf
Simulated futures from an ETS model
\end{block}\end{textblock}}

\setbeamertemplate{title page}
{\placefig{-0.01}{-0.01}{width=1.01\paperwidth,height=1.01\paperheight}{NYC.jpg}
\begin{textblock}{8.5}(.5,0.2)
  \raggedright\fontsize{20}{20}\selectfont\bfseries\sffamily\textcolor{Orange}{\inserttitle}\\[0.2cm]
  \fontsize{12}{12}\selectfont\bfseries\sffamily\textcolor{Orange}{\insertsubtitle}
  \end{textblock}
\placefig{.5}{6.2}{width=2.4cm}{tsibble.png}
\placefig{2.9}{6.2}{width=2.4cm}{feasts.png}
\placefig{5.3}{6.2}{width=2.4cm}{fable.png}
\begin{textblock}{7.5}(13.6,-.1)
  {\fontsize{9}{9}\sf\color{Orange}bit.ly/fable2023}
\end{textblock}
}

% Tikz plots
\usepackage{tikz}
\usetikzlibrary{trees,shapes,arrows,matrix,shadows,positioning,calc}
\tikzstyle{decision} = [diamond, draw, fill=blue!20,
    text width=4.5em, text badly centered, node distance=4cm, inner sep=0pt]
\tikzstyle{block} = [rectangle, draw, fill=blue!20,
    text width=5cm, text centered, rounded corners, minimum height=4em]
\tikzstyle{line} = [draw, thick, -latex']
\tikzstyle{cloud} = [draw, ellipse,fill=red!20, node distance=3cm,
    minimum height=2em, text centered]
\tikzstyle{connector} = [->,thick]

\tikzset{
  basic/.style  = {draw, text width=2cm, font=\sffamily, rectangle},
  root/.style   = {basic, text width=3cm, rounded corners=2pt, thin, align=center, fill=red!30},
  level 2a/.style = {basic, rounded corners=2pt, thin,align=center, fill=blue!50, text width=7em},
  level 2b/.style = {basic, rounded corners=2pt, thin,align=center, fill=green!50, text width=7em},
  level 3a/.style = {basic, rounded corners=2pt, thin, align=center, fill=blue!30, text width=4em},
  level 3b/.style = {basic, rounded corners=2pt, thin, align=center, fill=green!30, text width=4em},
  level 4a/.style = {basic, rounded corners=2pt, thin, align=left, fill=blue!10, text width=3.5em},
  level 4b/.style = {basic, rounded corners=2pt, thin, align=left, fill=green!10, text width=3.5em}
}

% % BIBLIOGRAPHIES
% \usepackage[style=authoryear,bibencoding=utf8,minnames=1,maxnames=4, maxbibnames=99,natbib=true,dashed=false,terseinits=true,giveninits=true,uniquename=false,uniquelist=false,labeldate=true,doi=false, isbn=false, natbib=true,backend=biber]{biblatex}

% \DeclareFieldFormat{url}{\url{#1}}
% \DeclareFieldFormat[article]{pages}{#1}
% \DeclareFieldFormat[inproceedings]{pages}{\lowercase{pp.}#1}
% \DeclareFieldFormat[incollection]{pages}{\lowercase{pp.}#1}
% \DeclareFieldFormat[article]{volume}{\mkbibbold{#1}}
% \DeclareFieldFormat[article]{number}{\mkbibparens{#1}}
% \DeclareFieldFormat[article]{title}{\MakeCapital{#1}}
% \DeclareFieldFormat[article]{url}{}
% \DeclareFieldFormat[Techreport]{Url}{}
% \DeclareFieldFormat[book]{url}{}
% \DeclareFieldFormat[inbook]{url}{}
% \DeclareFieldFormat[incollection]{url}{}
% \DeclareFieldFormat[inproceedings]{url}{}
% \DeclareFieldFormat[inproceedings]{title}{#1}
% \DeclareFieldFormat{shorthandwidth}{#1}
% %\DeclareFieldFormat{extrayear}{}
% % No dot before number of articles
% \usepackage{xpatch}
% \xpatchbibmacro{volume+number+eid}{\setunit*{\adddot}}{}{}{}
% % Remove In: for an article.
% \renewbibmacro{in:}{%
%   \ifentrytype{article}{}{%
%   \printtext{\bibstring{in}\intitlepunct}}}

% \AtEveryBibitem{\clearfield{month}}
% \AtEveryCitekey{\clearfield{month}}
% \AtBeginBibliography{\fontsize{11}{11}\sf}
\setbeamertemplate{frametitle continuation}{}
\makeatletter
\makeatother
\makeatletter
\makeatother
\makeatletter
\@ifpackageloaded{caption}{}{\usepackage{caption}}
\AtBeginDocument{%
\ifdefined\contentsname
  \renewcommand*\contentsname{Table of contents}
\else
  \newcommand\contentsname{Table of contents}
\fi
\ifdefined\listfigurename
  \renewcommand*\listfigurename{List of Figures}
\else
  \newcommand\listfigurename{List of Figures}
\fi
\ifdefined\listtablename
  \renewcommand*\listtablename{List of Tables}
\else
  \newcommand\listtablename{List of Tables}
\fi
\ifdefined\figurename
  \renewcommand*\figurename{Figure}
\else
  \newcommand\figurename{Figure}
\fi
\ifdefined\tablename
  \renewcommand*\tablename{Table}
\else
  \newcommand\tablename{Table}
\fi
}
\@ifpackageloaded{float}{}{\usepackage{float}}
\floatstyle{ruled}
\@ifundefined{c@chapter}{\newfloat{codelisting}{h}{lop}}{\newfloat{codelisting}{h}{lop}[chapter]}
\floatname{codelisting}{Listing}
\newcommand*\listoflistings{\listof{codelisting}{List of Listings}}
\makeatother
\makeatletter
\@ifpackageloaded{caption}{}{\usepackage{caption}}
\@ifpackageloaded{subcaption}{}{\usepackage{subcaption}}
\makeatother
\makeatletter
\makeatother
\ifLuaTeX
  \usepackage{selnolig}  % disable illegal ligatures
\fi
\IfFileExists{bookmark.sty}{\usepackage{bookmark}}{\usepackage{hyperref}}
\IfFileExists{xurl.sty}{\usepackage{xurl}}{} % add URL line breaks if available
\urlstyle{same} % disable monospaced font for URLs
\hypersetup{
  pdftitle={Africast-Tidy Time Series \& Forecasting Using R},
  hidelinks,
  pdfcreator={LaTeX via pandoc}}

\title{Africast-Tidy Time Series \& Forecasting Using R}
\author{}
\date{22 October 2023}

\begin{document}
\frame{\titlepage}
\begin{frame}{Instructors}
\protect\hypertarget{instructors}{}
\placefig{0.6}{1.5}{width=3.5cm}{Bahman}
\begin{textblock}{8.2}(5.6,1.3)
\begin{alertblock}{Bahman Rostami-Tabar}
\href{https://www.bahmanrt.com/}{\faIcon{home} bahmanrt.com}\\
\href{https://github.com/bahmanrostamitabar}{\faIcon{github}  @bahmanrostamitabar}\\
\href{mailto:Rostami-TabarB@cardiff.ac.uk}{\faIcon{envelope}  Rostami-TabarB@cardiff.ac.uk}
\end{alertblock}
\end{textblock}

\placefig{0.6}{5.4}{width=3.5cm}{Mitch}
\begin{textblock}{8.2}(5.6,5.4)
\begin{alertblock}{Mitchell O'Hara-Wild}
\href{https://mitchelloharawild.com}{\faIcon{home} mitchelloharawild.com}\\
\href{https://github.com/mitchoharawild}{\faIcon{github}  @mitchelloharawild}\\
\href{mailto:Mitch.OHara-Wild@monash.edu}{\faIcon{envelope}  Mitch.OHara-Wild@monash.edu}
\end{alertblock}
\end{textblock}
\end{frame}

\begin{frame}{Assumptions}
\protect\hypertarget{assumptions}{}
\begin{itemize}
\tightlist
\item
  This is not an introduction to R. We assume you are broadly
  comfortable with R code, the RStudio environment and the tidyverse.
\item
  This is not a statistics course. We assume you are familiar with
  concepts such as the mean, standard deviation, quantiles, regression,
  normal distribution, likelihood, etc.
\item
  This is not a theory course. We are not going to derive anything. We
  will teach you time series and forecasting tools, when to use them,
  and how to use them most effectively.
\end{itemize}
\end{frame}

\begin{frame}{Key reference}
\protect\hypertarget{key-reference}{}
\large

\begin{block}{}\bf
\hangafter=1\hangindent=.3cm
 {Hyndman, R.~J. \& Athanasopoulos, G. (2021) \emph{Forecasting: principles and practice}, 3rd ed.}
\end{block}\pause
\begin{alertblock}{}\Large
\centerline{\bf OTexts.org/fpp3/}
\end{alertblock}

\pause

\begin{itemize}
\tightlist
\item
  Free and online
\item
  Data sets in associated R package
\item
  R code for examples
\end{itemize}
\end{frame}

\begin{frame}{Reference - Recommended}
\protect\hypertarget{reference---recommended}{}
\large

\begin{block}{}\bf
\hangafter=1\hangindent=.3cm
 {Kolassa S., Rostami-Tabar B., \& Siemsen, E. (2023) \emph{https://dfep.netlify.app/}, 1st ed.}
\end{block}\pause
\begin{alertblock}{}\Large
\centerline{\bf https://dfep.netlify.app/}
\end{alertblock}

\pause
\fontsize{11}{11}\sf

\begin{itemize}
\tightlist
\item
  Free and online

  \begin{itemize}
  \tightlist
  \item
    Not an in-depth technical book
  \item
    Mindset behind forecasting
  \item
    Overview of forecasting methods and processes
  \end{itemize}
\end{itemize}
\end{frame}

\begin{frame}{International Institute of Forecasters}
\protect\hypertarget{international-institute-of-forecasters}{}
\begin{itemize}
\item
  a nonprofit organization founded in 1982, is dedicated to developing
  and furthering the generation, distribution, and use of knowledge on
  forecasting
\item
  \href{https://www.f4sg.org/}{Forecasting for Social Good}
\end{itemize}
\end{frame}

\begin{frame}{Poll: \href{https://www.menti.com/alfmobtvo9wb}{How
proficient are you in using R?}}
\protect\hypertarget{poll-how-proficient-are-you-in-using-r}{}
\fontsize{14}{15}\sf

\begin{enumerate}
\tightlist
\item
  Guru: The R core team come to me for advice.
\item
  Expert: I have written several packages on CRAN.
\item
  Skilled: I use it regularly and it is an important part of my job.
\item
  Comfortable: I use it often and am comfortable with the tool.
\item
  User: I use it sometimes, but I am often searching around for the
  right function.
\item
  Learner: I have used it a few times.
\item
  Beginner: I've managed to download and install it.
\item
  Unknown: Why are you speaking like a pirate?
\end{enumerate}
\end{frame}

\begin{frame}{Poll: \href{https://www.menti.com/alfmobtvo9wb}{How
experienced are you in forecasting}}
\protect\hypertarget{poll-how-experienced-are-you-in-forecasting}{}
\begin{enumerate}
\tightlist
\item
  Guru: I wrote the book, done it for decades, now I do the conference
  circuit.
\item
  Expert: It has been my full time job for more than a decade.
\item
  Skilled: I have been doing it for years.
\item
  Comfortable: I understand it and have done it.
\item
  Learner: I am still learning.
\item
  Beginner: I have heard of it and would like to learn more.
\item
  Unknown: What is forecasting? Is that what the weather people do?
\end{enumerate}
\end{frame}

\begin{frame}[fragile]{Install required packages}
\protect\hypertarget{install-required-packages}{}
\begin{Shaded}
\begin{Highlighting}[]
\FunctionTok{install.packages}\NormalTok{(}\FunctionTok{c}\NormalTok{(}
  \StringTok{"tidyverse"}\NormalTok{,}
  \StringTok{"fpp3"}\NormalTok{,}
  \StringTok{"GGally"}
\NormalTok{))}
\end{Highlighting}
\end{Shaded}
\end{frame}

\begin{frame}{Approximate outline}
\protect\hypertarget{approximate-outline}{}
\vspace*{-0.1cm}\centering\fontsize{12}{12}\sf
\begin{tabular}{rp{8.6cm}l}
  \bf Session & \bf Topic                   & \bf Chapter \\
  \midrule
  1       & 1. Basics of time series and data structures    & 2 \\
  1       & 2. Time series patterns and basic graphics      & 2 \\
  2       & 3. Transforming / adjusting time series         & 3 \\
  2       & 4. Computing and visualizing features     & 4 \\
  3       & 5. Basic modeling / forecasting & 1,3,5 \\
  3       & 6. Forecasting with regression         & 7,10 \\
  4       & 7. Exponential smoothing       & 8 \\
  4       & 8. ARIMA models                & 9 \\
  5       & 9. Basic training and test accuracy       & 5 \\
  5       & 10. Residual diagnostics and cross validation                & 5 \\
  \bottomrule
\end{tabular}

\vspace*{.0cm}\begin{alertblock}{}{\centerline{\Large\textbf{https://workshop.f4sg.org/africast/}}}
\end{alertblock}
\end{frame}

\begin{frame}{Access materials}
\protect\hypertarget{access-materials}{}
\begin{itemize}
\tightlist
\item
  AFRICAST workshop website
\item
  Slack for mentorship and Q/A
\end{itemize}
\end{frame}



\end{document}
